\documentclass[11pt]{article}
% Packages
\usepackage{amsmath}
\usepackage{fullpage}
\usepackage{amsfonts}
\usepackage{amsthm}
\usepackage{graphicx}
\usepackage{hyperref}
\usepackage{geometry}
\usepackage{fancyhdr}
\usepackage{cite}
\usepackage{algorithm}
\usepackage{algorithmic}
\usepackage{pgfplots}
\usepackage{lmodern}
\pgfplotsset{compat=1.18}

% Page layout
\geometry{a4paper, margin=1in}
\setlength{\headheight}{14.49998pt}
\setlength{\headsep}{0.1in}
% Header and Footer
\pagestyle{fancy}
\fancyhf{}
\fancyhead[C]{Shortest Path in Graphs}
\fancyhead[R]{\thepage}

% Title
\title{Shortest Path in Graphs}
\author{
    Xiangchen Tian \\
    \fontsize{13}{12}\selectfont
    IIIS, Tsinghua University \\
    \fontsize{13}{12}\selectfont
    \texttt{txc23@mails.tsinghua.edu.cn}
}
\date{}
\newcommand{\descendants}[1]{#1^\downarrow}
\newcommand{\ancestors}[1]{{#1}^\uparrow}
\newcommand{\cutval}[1]{{\cal C}(#1)}
\newcommand{\crossweight}[2]{{\cal C}(#1,#2)}
\theoremstyle{plain}
\newtheorem{lemma}{Lemma}[section]
\newtheorem{theorem}{Theorem}[section]
\newtheorem{definition}{Definition}[section]
\begin{document}
\maketitle
\begin{abstract}
  This paper explores algorithms for finding the shortest paths in graphs, focusing on both the 
  Single-Source Shortest Path (SSSP) problem and the All-Pairs Shortest Path (APSP) problem. 
  For the SSSP, we review Dijkstra's Algorithm for graphs with non-negative edge weights and 
  the Bellman-Ford Algorithm for graphs that may contain negative edge weights. For the APSP, 
  we discuss Johnson's Algorithm, which efficiently handles graphs with negative weights by 
  reweighting edges using feasible potentials and applying Dijkstra's Algorithm, and the 
  Floyd-Warshall Algorithm, a dynamic programming approach for computing shortest paths between 
  all pairs of vertices. Additionally, we explore the connection between min-sum products and 
  the APSP problem. We implemented these algorithms in C++ and tested them on random graphs to 
  compare their performance. The experimental results validate the theoretical analyses and 
  demonstrate the practical efficiency of the algorithms. 
  
  The source code is available on GitHub.

    \textbf{Keywords:} Graph Theory, Shortest Path
\end{abstract}

\tableofcontents

\newpage
\section{Introduction}
\subsection{Background and Outline}\

In graph theory, a very simple and basic idea is to ask: given a graph with a weight for each edge, 
find the path with the smallest weight between any two points in the graph. Here, the weight of a 
path is the sum of the weights of all the edges in the path. This paper first reviews the most basic 
solutions to this type of problem, then conducts a comprehensive analysis and comparison of the 
feasibility and running time of its variant algorithms, and finally tests the theory in practice by 
measuring the running time of the algorithm implemented in C++.


\section{Single-Source Shortest Path Problem(SSSP)}
\subsection{Problem description}\
Given a graph, we want to find a shortest path from a single source vertex $s$ to every other vertex 
in the graph. The graph can be directed or undirected, weighted or unweighted (just take all the 
weights equal to $1$, as a special case of the weighted graph), and can contain negative weights.

We first briefly review the case without negative weights. In this case, the algorithm is Dijkstra's 
Algorithm, which is similar to what is taught in class, except that the algorithm is more closely 
integrated with the actual implementation. We then briefly review the case with negative weights. 
In this case, the algorithm is the Bellman-Ford Algorithm, which can report the finding of negative 
cycles when there are (reachable) negative weight cycles, and give the optimal path for each point 
given the source point $s$ when there are no (reachable) negative cycles.

\subsection{Dijkstra's Algorithm for Non-negative Weights Graphs}\

Pesudo-code see Algorithm 1.

\begin{algorithm}
  \caption{Dijkstra's Algorithm}
  \begin{algorithmic}[1]
  \REQUIRE Digraph $G = (V, E)$ with edge-weights $w_e \geq 0$ and source vertex $s \in G$
  \ENSURE The shortest-path distances from each vertex to $s$
  \STATE add $s$ to heap with key $0$
  \FOR{$v \in V \setminus \{s\}$}
      \STATE add $v$ to heap with key $\infty$
  \ENDFOR
  \WHILE{heap not empty}
      \STATE $u \leftarrow$ deletemin
      \FOR{each neighbor $v$ of $u$}
          \STATE key($v$) $\leftarrow \min\{$key($v$), key($u$) + $w_{uv}\}$ \COMMENT{relax $uv$}
      \ENDFOR
  \ENDWHILE
  \end{algorithmic}
  \end{algorithm}

  Below we give a brief process for proving the correctness of the algorithm, which is similar to 
  the proof process in the textbook (Section 4.4, Greedy Algorithm) and the proof process described in class.

  \begin{lemma}
    Consider the set $V \backslash \{\text{current elements in the heap}\}$ at any point in the 
    algorithm's execution. For each vertices $u$ in $V \backslash \{\text{current elements in the heap}\}$, 
    the $\text{key}(u)$ of the priority queue is the shortest-path distances from each vertex to $s$.
  \end{lemma}

  \begin{proof}
    We prove the lemma by induction on the size of the set 
    \[
      S = V \backslash \{\text{current elements in the heap}\}
    \]

    The case $|S|=1$ is easy, since then we have $S = \left\{ s \right\} $ and $d(s) = 0$.

    Suppose the claim holds when $|S| = k$ for some value $k \geq 1$. We now grow $S$ to size $k+1$ by adding the node $v$.
    Let $e_{uv}=(u, v)$ be the final edge on our $s-v$ path $P_v$. Now consider any other $s-v$ path $P$, 
    this $P$ must leave the set $S$ somewhere to reach the vertex $v$, when it go out of $S$, it's accumulate 
    weight already larger than the weight of path $P_v$ (which is equal to $\text{key}(v)$), so $P_v$ is the shortest path from $s$ to $v$.
  \end{proof}

  Running time analysis: If we use a binary heap, which incurs $O(\log n)$ for decrease-key as well as extract-min 
  operations, we incur a running time of $O(m \log n)$. If we use a Fibonacci heap, which incurs $O(1)$ for decrease-key
  and $O(\log n)$ for extract-min operations, we incur a running time of $O(m + n \log n)$. These two methods are implemented in C++, see [TBD].

\subsection{Bellman-Ford Algorithm for Graphs with Negative Weights}\

Pesudo-code see Algorithm 2.

\begin{algorithm}
  \caption{Bellman-Ford Algorithm}
  \begin{algorithmic}[1]
  \REQUIRE A digraph $G = (V, E)$ with edge weights $w_e \in \mathbb{R}$, and source vertex $s \in V$
  \ENSURE The shortest-path distances from each vertex to $s$, or report that a negative-weight cycle exists
  \STATE $\text{dist}(s) \leftarrow 0$ \COMMENT{the source has distance 0}
  \FOR{each $v \in V$}
      \STATE $\text{dist}(v) \leftarrow \infty$
  \ENDFOR
  \FOR{$|V|$ iterations}
      \FOR{each edge $e = (u, v) \in E$}
          \STATE $\text{dist}(v) \leftarrow \min\{\text{dist}(v),\ \text{dist}(u) + \text{weight}(e)\}$
      \ENDFOR
  \ENDFOR
  \IF{any distances changed in the last (nth) iteration}
      \STATE output ``$G$ has a negative weight cycle''
  \ENDIF
  \end{algorithmic}
\end{algorithm}

  Below we also give a brief process for proving the correctness of the algorithm, which is similar to
  the proof process in the textbook (Section 6.8, Dynamic Programming) and the proof process described in class.
  In the discussion below, let $s$ be the source vertex, $t$ be the target vertex which is fixed. 
  This case is the opposite of the algorithm above, in which the source point is fixed and the sink point 
  will change; but despite this, the proof process is exactly the same. The simplest idea is to reverse each 
  directed edge and perform the same derivation, in which case the source point and sink point will be swapped.

  \begin{lemma}
    If $G$ has no negative cycles, then there is a shortest path from $s$ to $t$
    that is simple (i.e., does not repeat nodes), and hence has at most $n-1$ edges, where $n = |V|$.
  \end{lemma}

  \begin{proof}
    Since every cycle has nonnegative cost, the shortest path $P$ from $s$ to
    $t$ with the fewest number of edges does not repeat any vertex $v$. For if $P$ did
    repeat a vertex $v$, we could remove the portion of $P$ between consecutive visits
    to $v$, resulting in a path of no greater cost and fewer edges. There are at most 
    $n$ vertices, and $P$ does not repeat any vertex, so $P$ has at most $n-1$ edges.
  \end{proof}

  \begin{lemma}
    Denote $\text{OPT}(i, v)$ to be the minimum cost of a $v-t$ path using at most
    $i$ edges. Then if $i > 0$, we have
    \[
      \text{OPT}(i, v) =\min\left\{ \text{OPT}(i-1, v),\min_{w \in v\text{'s neighbour}}(\text{OPT}(i-1, w) + c_{vw}) \right\} 
    \]
  \end{lemma}

  \begin{proof}
    $\text{OPT}(i, v)$ can only have two possibilities: 
    \begin{itemize}
      \item[(i)] The path uses at most $i-1$ edges, in which case the cost is $\text{OPT}(i-1, v)$.
      \item[(ii)] The path uses exactly $i$ edges, in which case the cost is $\min_{w \in v\text{'s neighbour}}(\text{OPT}(i-1, w) + c_{vw})$.
    \end{itemize}
  \end{proof}

  Note: The algorithm provided above is the `Improving the Memory Requirements' version of the Bellman-Ford algorithm, 
  which only use $O(n)$ space. This version's correctness is based on the observation ``Throughout the algorithm $\text{dist}(v)$ 
  is the length of some path from $v$ to $t$, and after $i$ rounds of updates the value $\text{dist}(v)$ is no larger than the length of
  the shortest path from $v$ to $t$ using at most $i$ edges''. For details, see the KT textbook (pages 295 to 297).

  Running time analysis: there are $n = |V|$ iterations, and in each iteration, we visit all the edges, so the running time is $O(mn)$.

\section{All-Pairs Shortest Path Problem(APSP)}

  The obvious way to do this is to run an algorithm for SSSP $n$ times,
  each time with a different vertex being the source. This method will give $O(mn + n^{2} \log n)$
  runtime for non-negative weights graphs by using $n$ rounds of Dijkstra Algorithm, and $O(mn^{2})$ 
  runtime for graphs with negative weights by using $n$ rounds of Bellman-Ford Algorithm. But obviously, 
  we can take advantage of the correlation between $n$ problems from different source points to reduce 
  the time complexity to get a runtime of $O(mn + n^{2} \log n)$ with general edge weights, which is 
  known as Johnson's Algorithm.

\subsection{Feasible Potentials}\

  General idea: The negative weight is the main obstacle to the efficiency of the algorithm. Is there a way to
  transform the graph so that the weights are non-negative? The answer is yes, and this is why we introduce the
  concept of feasible potentials.

  \begin{definition}
    For a weighted digraph (a.k.a. directed graph) $G = (V, E)$, a function $\phi: V \rightarrow \mathbb{R}$ is
    a feasible potential if for all edges $e=(u, v) \in E$, 
    \[
      \phi(u) + w_{uv} - \phi(v) \geq 0
    \]
  \end{definition}

  This formula is a bit like the triangle inequality. (This reminds me of Admissible Heuristics in search algorithms 
  in reinforcement learning.) 

  Given a feasible potential, we can transform the edge-weights of the graph from $w_{uv}$ to 
  \[
    \hat w_{uv} = w_{uv} + \phi(u) - \phi(v)
  \]

  \begin{lemma}
    The new weights $\hat w$ are all positive.
  \end{lemma}

  \begin{proof}
    (Straight forward) Directly use the definition of the feasible potential.
  \end{proof}

  \begin{lemma}
    Let $P_{ab}$ be a path from vertex $a$ to vertex $b$. Let $\ell(P_{ab})$ be the length of the path $P_{ab}$
    when the edge weights are $w$. Let $\hat \ell(P_{ab})$ be the length of the path $P_{ab}$ when the edge weights
    are $\hat w$. Then 
    \[
      \hat \ell(P_{ab}) = \ell(P_{ab}) + \phi(a) - \phi(b)
    \]
  \end{lemma}

  \begin{proof}
    (Straight forward) The feasible potentials of all intermediate nodes cancel each other out.
  \end{proof}

  Lemma 3.2 tells us that if we can find a feasible potential, we can transform the graph so that the shortest path
  in the transformed graph is the same as the shortest path in the original graph and the new graph has non-negative weights.
  This is the key to Johnson's Algorithm. But how can we find feasible potentials? A very clever idea: Use Bellman-Ford Algorithm.
  suppose there some source vertex $s \in V$ such that every vertex in $V$ is reachable from $s$, if we set $\phi(v) = \text{dist}(s, v)$,
  then $\phi$ is a feasible potential because for any edge $e = (u, v)$, we have
  \[
    \phi(u) + w_{uv} - \phi(v) = \text{dist}(s, u) + w_{uv} - \text{dist}(s, v) \geq 0
  \]

  Intuition: The shortest path distance naturally satisfies the triangle inequality! The original graph can have negative weights,
  and we want to find a distance matric, obviously we should first run the Bellman-Fold algorithm once. We summarize the above discussion
  in the following lemma.

  \begin{lemma}
    Given a digraph $G = (V, A)$ with vertex $s$ such that all
    vertices are reachable from $s$, $\phi(v) = \text{dist}(s, v)$ is a feasible potential for $G$.
  \end{lemma}

  Some other properties of feasible potentials:
  \begin{lemma}
    If all edge-weights are non-negative, then $\phi(v) = 0$ is a feasible potential.
  \end{lemma}
  \begin{proof}
    (Straight forward) For all edges $e = (u, v)$, we have 
    \[
      \phi(u) + w_{uv} - \phi(v) = w_{uv} \geq 0
    \]
  \end{proof}

  \begin{lemma}
    Adding a constant to a feasible potential gives another feasible potential.
  \end{lemma}
  \begin{proof}
    (Straight forward) For all edges $e = (u, v)$, we have 
    \[
      \phi(u) + w_{uv} - \phi(v) = \phi(u) + c + w_{uv} - \phi(v) - c = \phi(u) + w_{uv} - \phi(v)
    \]
  \end{proof}

  \begin{lemma}
    If there is a negative cycle in the graph, there can be no feasible potential.
  \end{lemma}
  \begin{proof}
    The sum of the new weights along the cycle is
    the same as the sum of the original weights, due to the telescop-
    ing sum. But since the new weights are non-negative, so the old
    weight of the cycle must be, too, which is a contradiction.
  \end{proof}

  \begin{lemma}
    If we set the feasible potential of some vertex $s$ to $0$, i.e. $\phi(s) = 0$, then 
    $\phi(v)$ for any other vertex $v$ is an underestimate of the $s$-to-$v$ distance.
  \end{lemma}
  \begin{proof}
  For all the path from $s$ to $v$, we have
  \[
    0 \leq \hat \ell(P_{sv}) = \ell(P_{sv}) + \phi(s) - \phi(v) = \ell(P_{sv}) - \phi(v)
  \]
  \[
    \implies \ell(P_{sv}) \geq \phi(v)
  \]
  \end{proof}

  A result of Lemma 3.7 is that if we set the feasible potential of some vertex $s$ to $0$, 
  and try to maximize summation of $\phi(v)$ for other vertices subject to the feasible 
  potential constraints we will get an LP that is the dual of the shortest path LP:\@
  \begin{align*}
    \text{Maximize}\qquad &\sum_{x\in V} \phi(x) \\
    \text{Subject to}\qquad &\phi(s) = 0 \\
    &\phi(u) + w_{uv} - \phi(v) \geq 0 \quad \forall (u, v) \in E
  \end{align*}

\subsection{Johnson's Algorithm}\

Pesudo-code see Algorithm 3.

\begin{algorithm}
  \caption{Johnson's Algorithm}
  \begin{algorithmic}[1]
  \REQUIRE A weighted digraph $G = (V, A)$
  \ENSURE A list of the all-pairs shortest paths for $G$
  \STATE $V' \leftarrow V \cup \{s\}$ \COMMENT{add a new source vertex}
  \STATE $A' \leftarrow A \cup \{(s, v, 0) \mid v \in V\}$
  \STATE $\text{dist} \leftarrow \text{BellmanFord}((V', A'))$ \COMMENT{set feasible potentials}
  \FOR{each edge $e = (u, v) \in A$}
      \STATE $\text{weight}(e) \leftarrow \text{weight}(e) + \text{dist}(u) - \text{dist}(v)$
  \ENDFOR
  \STATE $L \leftarrow []$ \COMMENT{the result}
  \FOR{each $v \in V$}
      \STATE $L \leftarrow L + \text{Dijkstra}(G, v)$
  \ENDFOR
  \RETURN $L$
  \end{algorithmic}
\end{algorithm}

  The correctness of Johnson's Algorithm is discussed in Section 3.1. 

  Running time analysis: The Bellman-Ford algorithm takes $O(mn)$ time, the transformation of the weights takes $O(m)$ time,
  and the $n$ Dijkstra's algorithm takes $O(n(m + n \log n))$ time (if we use Fibonacci heaps), so the total running time is $O(mn + n^{2} \log n)$.

\subsection{Floyd-Warshall Algorithm}\

We introduce the Floyd-Warshall algorithm, which is a dynamic programming algorithm that solves the all-pairs shortest path problem
in a totally different perspective. 

Pesudo-code see Algorithm 4.

\begin{algorithm}
  \caption{Floyd-Warshall Algorithm}
  \begin{algorithmic}[1]
  \REQUIRE A weighted digraph $D = (V, E)$
  \ENSURE A list of the all-pairs shortest paths for $D$
  \FOR{each $(x, y) \in V \times V$}
      \IF{$(x, y) \in E$}
          \STATE $d(x, y) \leftarrow w_{xy}$
      \ELSE
          \STATE $d(x, y) \leftarrow \infty$
      \ENDIF
  \ENDFOR
  \FOR{each $z \in V$}
      \FOR{each $x \in V$}
          \FOR{each $y \in V$}
              \STATE $d(x, y) \leftarrow \min\{d(x, y), d(x, z) + d(z, y)\}$
          \ENDFOR
      \ENDFOR
  \ENDFOR
  \RETURN $d$
  \end{algorithmic}
\end{algorithm}

We prove the correctness of the Floyd-Warshall algorithm by induction on the number of vertices in the graph.

\begin{lemma}
  After we have considered vertices $V_k = \left\{ z_1, z_2, \cdots, z_k \right\} $ in the outer loop, 
  $\text{dist}(u, v)$ equals the weight of the shortest $x-y$ path that uses
  only the vertices from $V_k$ as internal vertices. (This is $\infty$ if there are no such
  paths.)
\end{lemma}

\begin{proof}
  We prove the lemma by induction on the number of vertices in $V_k$. (Dynamic programming)

  Base case: $k = 0$, the lemma is trivially true because the dist is initialized in that way.

  Inductive step: Suppose the lemma holds for $k = i$, we prove it for $k = i + 1$.

  Add $z_{i+1}$ to $V_i$, then the shortest path from $x$ to $y$ that uses only the vertices from $V_{i+1}$ as internal vertices
  is either the shortest path from $x$ to $y$ that uses only the vertices from $V_i$ as internal vertices, or the shortest path from $x$ to $z_{i+1}$
  that uses only the vertices from $V_i$ as internal vertices and the shortest path from $z_{i+1}$ to $y$ that also uses only the vertices from $V_i$ as internal vertices.

\end{proof}

Finally, when the outer loop is finished, the lemma is true for $k = n$. So the Floyd-Warshall algorithm is correct.

Running time analysis: The Floyd-Warshall algorithm takes $O(n^{3})$ time clearly. Although is not better than 
Johnson's algorithm, but is quite simple and easy to implement with minimal errors; also, the 
Floyd-Warshall algorithm can parellelizable, which is also quick in practice.

\subsection{Min-Sum Products and APSPs}\

A conceptually different way to get shortest-path algorithms is via matrix products. 
They seem to have nothing to do with each other, but if we modify the definition of 
matrix multiplication, we will find that there is a deep connection between them.

The standard definition of matrix multiplication is that
\[
  [A\cdot B]_{ij} = \sum_{k} A_{ik} \cdot B_{kj}
\]

where $[M]_{ij}$ is the element in the $i$-th row and $j$-th column of matrix $M$.
This definition can seen as matrix product on the field $(\mathbb{R}, +, \cdot)$.
We can surely change the underlying field to $(\mathbb{R}, \min, +)$, then we get the
Min-Sum Product (MSP):
\[
  [A\odot B]_{ij} = \min_{k} \left\{ A_{ik} + B_{kj} \right\} 
\]

i.e. above equation is the usual matrix multiplication, but over the semiring $(\mathbb{R}, \min, +)$.

It turns out that computing Min-Sum Products is precisely the operation needed for the APSP problem.
If we initialize a matrix $D$ as in the same way in Floyd-Warshall algorithm,
\[
  D_{ij} = \begin{cases}
    w_{ij} & \text{if } (i, j) \in E \\
    0 & \text{if } i = j \\
    \infty & \text{otherwise}
  \end{cases}
\]

Then consider the meaning of $[D \odot D]_{ij}$,
\[
  [D \odot D]_{ij} = \min_{k} \left\{ D_{ik} + D_{kj} \right\}
\]

This is exactly the meaning of the shortest path from $i$ to $j$ that uses only one intermediate vertex!
Similarly, $[D \odot D \odot D]_{ij}$ is the shortest path from $i$ to $j$ that uses only two intermediate vertices, and so on.
Since the shortest paths would have at most $n-2$ intermediate vertices, we only need to compute $D^{\odot n-1}$.

If we calculate the Min-Sum Products of the matrix one by one, each multiplication will take $O(n^3)$ time, 
a total of $O(n)$ multiplications will be required, and a total of $O(n^4)$ time will be required, which is 
slower than the previous algorithms. A simple improve is noticed that 
\[
  D^{\odot 2k} = D^{\odot k}\odot D^{\odot k}
\]
then the time complexity is reduced to $O(n^3 \log n)$.

\section{Experiment Results}

I implemented all the algorithm above in C++, and tested them on a random graph with $n$ vertices and $m$ edges.
The code is available on Github.

\bibliographystyle{plain} % or any other style you prefer
\bibliography{references} % use the .bib file
\end{document}